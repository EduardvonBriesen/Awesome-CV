%!TEX TS-program = xelatex
%!TEX encoding = UTF-8 Unicode
% Awesome CV LaTeX Template for Cover Letter
%
% This template has been downloaded from:
% https://github.com/posquit0/Awesome-CV
%
% Authors:
% Claud D. Park <posquit0.bj@gmail.com>
% Lars Richter <mail@ayeks.de>
%
% Template license:
% CC BY-SA 4.0 (https://creativecommons.org/licenses/by-sa/4.0/)
%


%-------------------------------------------------------------------------------
% CONFIGURATIONS
%-------------------------------------------------------------------------------
% A4 paper size by default, use 'letterpaper' for US letter
\documentclass[11pt, a4paper]{awesome-cv}
\usepackage{csquotes}
\usepackage[german]{babel}
\usepackage{amsmath}
\usepackage{unicode-math}
\usepackage{tabularx,colortbl} 

% Configure page margins with geometry
\geometry{left=2cm, top=1.8cm, right=2cm, bottom=1.8cm, footskip=.5cm}

% Specify the location of the included fonts
\fontdir[fonts/]

% Color for highlights
% Awesome Colors: awesome-emerald, awesome-skyblue, awesome-red, awesome-pink, awesome-orange
%                 awesome-nephritis, awesome-concrete, awesome-darknight
% \colorlet{awesome}{awesome-emerald}
% Uncomment if you would like to specify your own color
\definecolor{awesome}{HTML}{00883A}

% Colors for text
% Uncomment if you would like to specify your own color
% \definecolor{darktext}{HTML}{414141}
% \definecolor{text}{HTML}{333333}
% \definecolor{graytext}{HTML}{5D5D5D}
% \definecolor{lighttext}{HTML}{999999}

% Set false if you don't want to highlight section with awesome color
\setbool{acvSectionColorHighlight}{true}

% If you would like to change the social information separator from a pipe (|) to something else
\renewcommand{\acvHeaderSocialSep}{\quad\textbar\quad}


%-------------------------------------------------------------------------------
%	PERSONAL INFORMATION
%	Comment any of the lines below if they are not required
%-------------------------------------------------------------------------------
% Available options: circle|rectangle,edge/noedge,left/right
\photo[circle,noedge,left]{bewerbungsfoto-edit}
\name{Eduard}{von Briesen}
\position{B.Sc. Informatik}
\address{Am Hang 1, 85567 Grafing}

\mobile{(+49) 1575 7334727}
\email{e.v.briesen@gmail.com}
%\homepage{www.posquit0.com}
\github{EduardvonBriesen}
%\linkedin{Niklas Mamtschur}
% \gitlab{gitlab-id}
% \stackoverflow{SO-id}{SO-name}
% \twitter{@twit}
% \skype{skype-id}
% \reddit{reddit-id}
% \medium{madium-id}
% \googlescholar{googlescholar-id}{name-to-display}
%% \firstname and \lastname will be used
% \googlescholar{googlescholar-id}{}
% \extrainfo{extra informations}

%\quote{``Be the change that you want to see in the world."}


%-------------------------------------------------------------------------------
%	LETTER INFORMATION
%	All of the below lines must be filled out
%-------------------------------------------------------------------------------
% The company being applied to
\recipient
 {Lehr- und Forschungseinheit Medieninformatik }
 {Ludwig-Maximilians-Universität München}
% The date on the letter, default is the date of compilation
\letterdate{\today}
% The title of the letter
\lettertitle{Motivationsschreiben - Master Mensch-Computer-Interaktion}
% How the letter is opened
\letteropening{Sehr geehrte Auswahlkommission,}
% How the letter is closed
\letterclosing{Mit freundlichen Grüßen,}
% Any enclosures with the letter
% \letterenclosure[Angehängt]{Lebenslauf}


%-------------------------------------------------------------------------------
\begin{document}

% Print the header with above personal informations
% Give optional argument to change alignment(C: center, L: left, R: right)
\makecvheader[R]

% Print the footer with 3 arguments(<left>, <center>, <right>)
% Leave any of these blank if they are not needed
\makecvfooter
{\today}
{Motivationsschreiben}
{Eduard von Briesen}

% Print the title with above letter informations
\makelettertitle

%-------------------------------------------------------------------------------
%	LETTER CONTENT
%-------------------------------------------------------------------------------
\begin{cvletter}

  nach meinem erfolgreich abgeschlossenem Bachelor Studium der Informatik an der TU München, möchte ich meine erlangten Kenntnisse nun in dem Studiengang Mensch-Computer-Interaktion im Master ausbauen und vertiefen.

  \lettersection{Studium}

  Ich setze mich nun schon seit einigen Jahren mit Informatik in all ihren Facetten, sowohl in der akademischen wie auch der beruflichen Welt, auseinander.
  Begonnen hat diese Faszination schon mit dem ersten Kontaktpunkten in meiner Schulzeit, seitdem stand für mich fest, ich möchte eine Karriere in diesem Fachgebiet verfolgen.
  Mit der unglaublichen Vielfalt dieser Disziplin wurde ich dann in den ersten Semestern meines Bachelorstudiums konfrontiert.


  Ich eignete mir schnell die nötigen Grundkenntnisse an und auch wenn das ein oder andere Fach mein Können durchaus stark forderte, fand ich keines, für das ich mich nicht begeistern konnte.
  Dazu zählen Kompetenzen in den Kernbereichen der Informatik wie z.B.: \textit{Algorithmen und Datenstrukturen, Rechnerarchitektur, Betriebssysteme, Theoretische Informatik, Datenbanken,} sowie \textit{Rechnernetze und verteilte Systeme}.
  Zudem setze mein Studium einen großen Wert auf die dahinterliegenden mathematischen Grundlagen wie z.B.: \textit{Lineare Algebra, Analysis, Diskrete Strukturen, Diskrete Wahrscheinlichkeitstheorie} und \textit{Numerisches Programmieren}.
  Eine detaillierte Auflistung und Gegenüberstellung meiner belegten Grundlagenvorlesungen und der Module aus dem Bachelorstudiengang Mensch-Maschine Interaktion, finden sie im Anhang.


  Neben den genannten theoretischen Bereichen der Informatik konnte ich meine Kompetenzen im Programmieren ebenso in praxisnahen Projekten ausbauen.
  So erlangte ich ausgeprägte Kenntnisse in objektorientierten Sprachen wie \textit{Java}, Low-Level Sprachen wie \textit{x86 Assembler} oder \textit{C}, sowie funktionalen Sprachen wie \textit{Haskell}.


  Im weiteren Verlauf meines Studiums vertiefte ich mich in verschiedenste Felder der Informatik wie Künstliche Intelligenz oder Muster in der Softwaretechnik, immer auf der Suche nach neuen hochspannenden Konzepten. Im Rahmen meines Anwendungsfachs lernte ich die Intrigen der Wirtschaftswissenschaften kennen und fand große Freude daran mein Wissen auch interdisziplinär auszubauen. Besonders begeisterten mich auch überfachliche Grundlagen wie Ethik, IT-Recht und Soziologie.


  Zum Ende meines Studiums stellte sich wissenschaftliches Arbeiten als letzte fordernde Disziplin heraus. Sowohl selbständig im Rahmen meines Bachelor-Seminars: \textit{Security und Verification}, als auch im Team für mein Bachelor Praktikum: \textit{Computer Systems Lab} konnte ich mir die notwendigen Fahigkeiten aneignen.


  Kulminiert hat all das bisher genannte fachliche Know-How in meiner Bachelor Thesis:  \textit{Lambda-CNTR: A Dependable Serverless Architecture}, einem Tool zum Debugging von Serverless Applikationen in einem Kubernetes Umfeld. Hier eignete ich mir unter anderem die Sprachen \textit{Python} und \textit{Rust} an und gewann Einblicke in Technologien am bleeding Edge im Bereich der verteilten Systeme, wie die in meiner Arbeit zentralen Lambda-Funktionen.
  Der Lehrstuhl legte großen Wert auf ein sauberes wissenschaftliches Arbeiten, mit dem Ziel einer Publikation meiner Arbeit in Kombination mit einer themennahen Master Thesis.


  \lettersection{Berufserfahrung}

  Sobald ich meine Leidenschaft für Informatik entdeckt hatte fing ich bereits an, in das Berufsleben hineinzuschnuppern. Anfangs in Form eines zweiwöchigen Praktika während meiner Schulzeit bei dem lokalen IT-Dienstleister \textit{DIE SOFTWARE}, wo ich einige Intranet Seiten überarbeiten durfte. Hier kam ich erstmals mit der Entwicklung im Frontend in Kontakt und konnte in der doch kurzen Zeit einiges über das Arbeitsumfeld im IT-Bereich lernen.

  Einen komplett anderen, aber ebenso interessanten Einblick in die Arbeitswelt der Informatik gewann ich während eines Praktikum bei \textit{BMW}. Dort hatte ich das Vergnügen einem Bereichsleiter zu beschatten und dessen Alltag aus Kundengesprächen, Scrum-Meetings und auch Ausflügen in Werke mitzuerleben. Als damals noch angehender Informatik Student waren mir ein Großteil der gewonnen Eindrücke noch unbegreiflich, ich wurde aber im Laufe meines Studiums überraschend oft an diese Zeit erinnert, insbesondere in Bereichen des Software Engineerings.

  Im Oktober 2020 fing ich als Werkstudent bei der \textit{adesso SE} an. Hier konnte ich meine, aus dem Studium mitgenommenen, Fachkenntnisse nun endlich auch praxisnah zum Einsatz zu bringen. Ich merkte schnell, dass mir an vielen Stellen mein theoretisches Wissen nicht immer weiterhalf und eignete mir in kürzester Zeit, Vieles des von ein Fullstack Entwickler erwartete Know-How, an. Dazu gehörten v.A. ein kompetenter Umgang mit Versionierungssoftware wie \textit{Git}, Java-Frameworks wie \textit{SpringBoot}, Frontend Technologien wie \textit{Angular}, DevOps Prozessen für CI/CD und vor allem der kollaborativen Arbeit im Team.

  Ich hatte das große Glück zusammen mit gleichgesinnten Studenten an einem Research Projekt arbeiten zu können, an dem wir bis heute immer noch weiterentwickeln. Trotz der Pandemie und der ersten Monate im Home-Office entstand ein fantastischer Teamgeist und viele neue Freundschaften. Ziel unseres Teams ist die Entwicklung eines Log-Data-Analyser für moderne verteilte Systeme. In der ersten Phase des Projekts bauten wir unsere eigene Application auf Basis einer Microservice Architektur um ein besseres Verständnis für die unterliegenden Technologien zu gewinnen. Zu den eingesetzten Technologien zählen \textit{SpringBoot, Angular, Docker, Kubernetes, ELK-Stack, Jenkins,} uvm..

  Selten habe ich in meinem Studium eine solche Dichte an neuen Informationen aufgenommen und diese dann auch direkt zum Einsatz bringen können. Im Rahmen dieses Projekt entdeckte ich auch meine Vorliebe für das Entwickel und Designen von Benutzeroberflächen.

  \lettersection{Soziales}

  Mit zu meinen bevorzugten Freizeitaktivitäten zählt der Sport. Ich mache nun schon seit etwa 16 Jahren Judo und bin etwa genau so lange auch in meinem lokalen Verein aktiv. Dort gestalte ich als lizenzierter Trainer den Trainingsbetrieb mit und bin fester Bestandteil unserer Liga-Mannschaft. Seit kurzem bin ich auch Vorstand unseres Fördervereins und bin damit für die Förderung unserer Jugend z.B. durch das Organisieren von Freizeitaktionen verantwortlich.

  In meiner eigenen Zeit tüftele ich gerne an Projekten, wie den Bau eines 3D-Druckers, dem Umfunktionieren alter Röhrenmonitore zu Oszilloskopen oder rudimentärer Musikproduktion. So kann ich mich neben meinem Studium auch handwerklich und vor allem kreativ austoben. In Gesellschaft erlebe ich am liebsten Kultur in jeder Form, dazu zählen Museumsbesuche, Film und Theater genauso, wie Konzerte oder Poetry Slams.

  \lettersection{Motivation}

  Auf den Masterstudiengang Mensch-Computer-Interaktion bin ich durch eine Kollegin bzw. Kommilitonin aufmerksam geworden und mein Interesse wurde sofort geweckt. Besonders begeistert war ich von der Möglichkeit, meine weitreichenden aber sehr technischen Kenntnisse der Informatik, in einem kreativen und schöpferischen Umfeld einzubringen.

  % Solange ich mich mit Software beschäftige, habe ich ein tiefe Wertschätzung für ein intuitives und attraktives Design von Benutzeroberflächen. Soviel ich mich auch mit den dahinter laufenden Technologien auseinander setzte, blieben User Interfaces immer das gängigste Mittel der Interaktion. Deshalb bin ich sicher, dass ich fehlende Kompetenzen aus Bereichen der Medieninformatik wissbegierig aufgreifen werde, um schnell den Anschluss an weiterführende Fächer zu finden.

  Mit meinem erfolgreich abgeschlossenem Bachelorstudium, sowie meiner zahlreichen praktischen Erfahrungen, bin ich zuversichtlich, meine Qualifikation für den Masterstudiengang Mensch-Computer-Interaktion, belegt zu haben.

  \makeletterclosing

\end{cvletter}


%-------------------------------------------------------------------------------
% Print the signature and enclosures with above letter informations

\newpage


\begin{table*}
  \scriptsize
  \centering
  \begin{tabularx}{\textwidth}{|X|c|X|c|c|}

    \hline
    Module - LMU Bachelor Mensch-Maschine-Interaktion      & ECTS & Module - TUM Bachelor Informatik               & ECTS & Note \\
    \hline
    Einführung in die Programmierung                       & 9    & Einführung in die Informatik 1                 & 6    & 3,3  \\
                                                           &      & Einführung in die Informatik 2                 & 5    & 3,0  \\

    \rowcolor{awesome!25}
    Analysis für Informatiker und Statistiker              & 9    & Analysis für Informatik                        & 8    & 4,0  \\

    Programmierung und Modellierung                        & 6    & Grundlagenpraktikum: Programmierung            & 6    & 3,3  \\

    \rowcolor{awesome!25}
    Algorithmen und Datenstrukturen                        & 6    & Grundlagen: Algorithmen und Datenstrukturen    & 6    & 3,7  \\

    Rechnerarchitektur                                     & 6    & Einführung in die Rechnerarchitektur           & 8    & 3,0  \\

    \rowcolor{awesome!25}
    Softwareenticklungspraktikum oder Systempraktikum      & 12   & Praktikum: Computer Systems Lab                & 10   & 2,3  \\
    \rowcolor{awesome!25}
                                                           &      & Rechnerarchitektur-Praktikum                   & 8    & 3,7  \\

    Lineare Algebra für Informatiker                       & 6    & Lineare Algebra für Informatik                 & 8    & 4,0  \\

    \rowcolor{awesome!25}
    Statistik I für Studierende der Medieninformatik       & 6    & Diskrete Strukturen                            & 8    & 2,3  \\
    \rowcolor{awesome!25}
                                                           &      & Diskrete Wahrscheinlichkeitstheorie            & 6    & 3,3  \\

    Rechnernetze und verteilte Systeme                     & 6    & Grundlagen: Rechnernetze und Verteilte Systeme & 6    & 3,3  \\

    \rowcolor{awesome!25}
    Formale Sprachen und Komplexität /                     & 3    & Einführung in die Theoretische Informatik      & 8    & 4,0  \\
    \rowcolor{awesome!25}
    Theoretische Informatik für MI                         &      &                                                &      &      \\

    Seminar zu ausgewählten Themen der Informatik          & 3    & Seminar - Security und Verification            & 5    & 3,0  \\

    \rowcolor{awesome!25}
    Vertiefende Themen der Medieninformatik für Bachelor 1 & 6    & Algorithmic Game Theory                        & 5    & 3,3  \\

    Vertiefende Themen der Medieninformatik für Bachelor 2 & 6    & Muster in der Softwaretechnik                  & 5    & 3,7  \\

    \rowcolor{awesome!25}
    Vertiefende Themen der Medieninformatik für Bachelor 3 & 6    & Grundlagen der Künstlichen Intelligenz         & 5    & 2,7  \\

    Softwaretechnik                                        & 6    & Einführung in die Softwaretechnik              & 6    & 2,7  \\

    \rowcolor{awesome!25}
    Datenbanksysteme                                       & 6    & Grundlagen: Datenbanken                        & 6    & 2,3  \\

    Betriebssysteme                                        & 6    & Grundlagen: Betriebssysteme und Systemsoftware & 6    & 2,7  \\

    \rowcolor{awesome!25}
    Ethik und Recht in der Informatik                      & 3    & IT-Recht in der öffentlichen Verwaltung        & 5    & 1,7  \\
    \rowcolor{awesome!25}
                                                           &      & Seminar Wissenschaftler und Ethik              & 4    & 1,3  \\

    Bachelorarbeit                                         & 12   & Bachelor's Thesis                              & 12   & 2,3  \\

    \rowcolor{awesome!25}
    Disputation                                            & 3    & Bachelor-Kolloquium                            & 3    & 2,3  \\
    \hline
  \end{tabularx}
  \label{table:1}
  \caption{Gegenüberstellung der Module des Bachelor Studiengang Mensch-Maschine Interaktion und vergleichbaren Modulen aus meinem Bachelorstudium.}
\end{table*}

\begin{table*}
  \scriptsize
  \centering
  \begin{tabularx}{.5\textwidth}{|X|c|}

    \hline
    Module - LMU Bachelor Mensch-Maschine-Interaktion & ECTS  \\
    \hline

    Digitale Medien                                   & 9     \\

    \rowcolor{awesome!25}
    Grundbegriffe der Psychologie 1 + 2               & 2+2   \\

    Medientechnik                                     & 6     \\

    \rowcolor{awesome!25}
    Teilnahme an Benutzerstudien (6 VP-Stunden)       & 6     \\

    UX 1 - 3                                          & 6+6+6 \\

    \rowcolor{awesome!25}
    Computergrafik                                    & 6     \\

    Soziale und persönliche Kompetenz                 & 3     \\

    \rowcolor{awesome!25}
    Projektkompetenz Multimedia                       & 3     \\
    \hline
  \end{tabularx}
  \label{table:2}
  \caption{Fehlende Module aus dem Bachelor Studiengang Mensch-Maschine Interaktion.}
\end{table*}

\begin{table*}
  \scriptsize
  \centering
  \begin{tabularx}{.5\textwidth}{|X|c|c|}
    \hline
    Module - TUM Bachelor Informatik              & ECTS & Note \\
    \hline

    Numerisches Programmieren                     & 6    & 2,7  \\

    \rowcolor{awesome!25}
    Grundlagen der Betriebswirtschaftslehre 1 + 2 & 3+3  & 3,5  \\

    Kostenrechnung                                & 6    & 3,3  \\

    \rowcolor{awesome!25}
    Arbeits- und Industriesoziologie              & 3    & 1,7  \\

    Economics I: Microeconomics                   & 6    & 4,0  \\
    \hline
  \end{tabularx}
  \label{table:2}
  \caption{Belegte Module ohne direktes Pendent im Bachelor Studiengang Mensch-Maschine Interaktion.}
\end{table*}

\end{document}
